\documentclass[a4paper,12pt]{article}
\usepackage[utf8]{inputenc}
\usepackage[T1]{fontenc}
\usepackage[slovene]{babel}
\usepackage{lmodern}  
\usepackage{amsmath,amssymb}
\usepackage{booktabs}
\usepackage{graphicx}
\usepackage{float}

\begin{document}

\begin{titlepage}
    \centering
    \textsc{Fakulteta za matematiko in fiziko\\Univerza v Ljubljani}\\[25pt]
    
    \vspace{3cm}
    
    {\Huge \textbf{Zunajravninski grafi}\par}
    \vspace{1cm}
    {\Large Poročilo naloge\par}

    \vfill  
    {\large Jure Kraševec, Urh Videčnik\par}
    \vspace{1cm}
    {\large Februar 2026\par}
\end{titlepage}

\tableofcontents
\newpage

\section{Teoretični uvod}
Naj bo $G = (V, E)$ graf, kjer je $V$ množica vozlišč in $E$ množica povezav med vozlišči. Zunajravninski graf (ang. Outherplanar graph) je graf, 
ki ga lahko narišemo v ravnini tako, da se nobeni dve povezavi ne sekata in da vsa vozlišča ležijo na zunanji strani oziroma zunanjem licu grafa. 
Takšni grafi so podmnožica ravninskih grafov.

\noindent 
Množica vozlišč $S \subseteq V$ grafa $G$ je liha neodvisna množica, če za vsako vozlišče $u \in I$ velja, 
da je lihe stopnje ter za vsak par $u, v \in I $ velja, da med njima ne obstaja povezava v $E$. Za liho neodvisno množico $S$ velja, da vsako vozlišče 
$v \in V \setminus S$ nima nobenega soseda v $S$: $N(v) \cap S = \emptyset$ ali pa ima liho število sosedov v $S$: $|N(v) \cap S| \equiv 1 \pmod{2}$.
Z $\alpha_{od}(G)$ označimo največjo moč lihe neodvisne množice grafa $G$.

\section{Cilj naloge}
Cilj naloge je preveriti, veljavnost neenakosti
$$\alpha_{od}(G) \geqslant n/7,$$
za vsak zunajravninski graf $G$ z $n$ vozlišči.
Nalogo smo razdelili na dva dela. Prvi del je namenjen preverjanju neenakosti na manjših zunajravninskih grafih - grafih z največ $10$ vozlišči.
V tem delu bomo implementirali algoritem, ki bo generiral vse zunajravninske grafe z največ $10$ vozlišči in za vsak graf izračunal $\alpha_{od}(G)$.
V drugem delu pa bomo s postopkom lepljenja zunajravninskih grafov, generirali zunajravninske grafe z več kot $10$ vozlišči in
 poskušali najti protiprimer, torej graf, za katerega ne velja zgornja neenakost.

\section{Zunajravninski grafi na $ n \leqslant 10$ vozliščih}
Najprej bomo preverili neenakost $\alpha_{od}(G) \geqslant n/7$ za vse zunajravninske grafe z največ $10$ vozlišči.
Prvo bomo implementirali funkcijo, ki za dan graf preveri, ali je zunajravninski. Ker za vsak zunajravninski graf velja, 
da ga lahko narišemo kot krožno ravninski graf (angl. Circular planar graph), lahko za preverjanje zunajravninskosti uporabimo 
vgrajeno metodo \texttt{is\_circular\_planar}. Funkcijo za preverjanje zunajravninskosti smo testirali na nekaj grafih, da smo se prepričali o njeni pravilnosti.





\end{document}