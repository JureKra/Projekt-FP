\documentclass[a4paper,12pt]{article}
\usepackage[utf8]{inputenc}
\usepackage[T1]{fontenc}
\usepackage[slovene]{babel}
\usepackage{lmodern}  
\usepackage{amsmath,amssymb}
\usepackage{booktabs}
\usepackage{graphicx}
\usepackage{float}

\newcommand{\fn}[1]{\texttt{#1}}

\begin{document}

\begin{titlepage}
    \centering
    \textsc{Fakulteta za matematiko in fiziko\\Univerza v Ljubljani}\\[25pt]
    {\Large Univerza v Ljubljani\\ Fakulteta za matematiko in fiziko\par}
    
    \vspace{3cm}
    
    {\Huge \textbf{Zunajravninski grafi}\par}
    
    \vspace{3cm}
    
    {\large Avtorja:\par}
    \vspace{0.2cm}
    {\large Jure Kraševec\\ Urh Videčnik\par}
    
    \vfill
    
    {\large Februar 2026\par}
\end{titlepage}

\tableofcontents
\newpage

\section{Teoretičen uvod}
Naj bo $G = (V, E)$ graf, kjer je $V$ množica vozlišč in $E$ množica povezav med vozlišči. Zunajravninski graf (ang. Outherplanar graph) je graf, 
ki ga lahko narišemo v ravnini tako, da se nobeni dve povezavi ne sekata in da vsa vozlišča ležijo na zunanji strani oziroma zunanjem licu grafa. 
Takšni grafi so podmnožica ravninskih grafov.

\noindent 
Množica vozlišč $S \subseteq V$ grafa $G$ je liha neodvisna množica, če za vsako vozlišče $u \in I$ velja, 
da je lihe stopnje ter za vsak par $u, v \in I $ velja, da med njima ne obstaja povezava v $E$. Za liho neodvisno množico $S$ velja, da vsako vozlišče 
$v \in V \setminus S$ nima nobenega soseda v $S$: $N(v) \cap S = \emptyset$ ali pa ima liho število sosedov v $S$: $|N(v) \cap S| \equiv 1 \pmod{2}$.
Z $\alpha_{od}(G)$ označimo največjo moč lihe neodvisne množice grafa $G$.

\section{Cilj naloge}
Cilj naloge je preveriti, če za vsak outerplanaren graf $G$ velja neenakost
$$\alpha_{od}(G) \geqslant n/7,$$.


\section{Zunajravninski grafi na $ n \leqslant 10$ vozliščih}


\end{document}