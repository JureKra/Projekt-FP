\documentclass{article}
\usepackage{amsmath}
\usepackage{amsthm}
\usepackage{amssymb}
\usepackage[slovene]{babel}
\usepackage[utf8]{inputenc}
\usepackage[T1]{fontenc}

\title{Outerplanar graphs}
\author{Jure Kraševec, Urh Videčnik}

\begin{document}
\maketitle

\section{Uvod}
Naj bo $G = (V, E)$ graf, kjer je $V$ množica vozlišč in $E$ množica povezav med vozlišči. Outerplanarni graf (ang. Outherplanar graph) je graf, ki ga lahko narišemo v ravnini tako, 
da se nobeni dve povezavi ne sekata in da vsa vozlišča ležijo na zunanji strani (licu) grafa. Takšni grafi so podmnožica 
planarnih oziroma ravninskih grafov, ki jih lahko narišemo v ravnini brez presekajočih se povezav, 
vendar vozlišča niso nujno na zunanji strani. Graf je zunajplanaren natanko tedaj, ko ne vsebuje podgrafa, 
ki je homeomorfen z $K_4$ (poln graf na štirih vozliščih) ali z $K_{2,3}$ (poln bipartiten graf z dvema in tremi vozlišči v obeh delih). 

\noindent 
Množica vozlišč $I \subseteq V$ grafa $G$ je liha neodvisna množica, če za vsako vozlišče $u \in I$ velja, da je lihe stopnje ter za vsak par $u, v \in I $ velja, da med njima ne obstaja povezava v $E$.



\section{Opis problema}
V nalogi nas zanima, če za vsak zunajplanarni graf $G$ velja neenakost $\alpha_{od}(G) \geqslant n/7$, 
kjer je $n$ število vozlišč grafa $G$. 

\section{Načrt dela}



\end{document}