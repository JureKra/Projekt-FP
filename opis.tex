\documentclass{article}
\usepackage{amsmath}
\usepackage{amsthm}
\usepackage{amssymb}
\usepackage[slovene]{babel}
\usepackage[utf8]{inputenc}
\usepackage[T1]{fontenc}

\title{Outerplanarni grafi}
\author{Jure Kraševec, Urh Videčnik}

\begin{document}
\maketitle

\section{Uvod}
Naj bo $G = (V, E)$ graf, kjer je $V$ množica vozlišč in $E$ množica povezav med vozlišči. Outerplanarni graf (ang. Outherplanar graph) je graf, ki ga lahko narišemo v ravnini tako, 
da se nobeni dve povezavi ne sekata in da vsa vozlišča ležijo na zunanji strani oziroma zunanjem licu grafa. Takšni grafi so podmnožica 
planarnih oziroma ravninskih grafov, ki jih lahko narišemo v ravnini brez presekajočih se povezav, 
vendar vozlišča niso nujno na zunanji strani. Graf je outerplanaren natanko tedaj, ko ne vsebuje podgrafa, 
ki je homeomorfen z $K_4$ ali z $K_{2,3}$. 

\noindent 
Množica vozlišč $S \subseteq V$ grafa $G$ je liha neodvisna množica, če za vsako vozlišče $u \in I$ velja, da je lihe stopnje ter za vsak par $u, v \in I $ velja, da med njima ne obstaja povezava v $E$.
Poleg tega velja, da vsako vozlišče $v \in V \setminus S$ nima nobenega soseda v $S$: $N(v) \cap S = \emptyset$ ali pa ima liho število sosedov v $S$: $|N(v) \cap S| \equiv 1 \pmod{2}$ je liho število.
$N(v)$ označuje množico sosedov vozlišča $v$. Največji moči takšne množice pravimo liho neodvisno število grafa, ki ga označimo z $\alpha_{od}(G)$.


\section{Opredelitev problema}
V nalogi nas bo zanimalo število $\alpha_{od}(G)$. Cilj naloge, je preveriti, če za vsak outerplanaren graf $G$ velja neenakost 
$$\alpha_{od}(G) \geqslant n/7,$$ 
kjer je $n$ število vozlišč grafa $G$. 
Najprej bo potrebno implementirati funkcijo, ki za dan graf preveri, če je outerplanaren. 
Nato bo potrebno razviti algoritem, ki za dan outerplanaren graf izračuna liho neodvisno število $\alpha_{od}(G)$ 
ter preveri, če velja neenakost $\alpha_{od}(G) \geqslant n/7$.
V nadaljevanju bo treba implementirati funkcijo, ki bo generirala naključne outerplanarne grafe različnih velikosti in zanje preverila zgornjo neenakost.
Večje outerplanarne grafe bomo generirali z združevanjem ciklov poljubne dolžine, katerim dodajamo nepresečne diagonale. Dobljene grafe skupaj zlepimo
v drevesno strukturo.

\section{Načrt dela}

Skupno delo bo potekalo prek GitHub-a in v programskem okolju \texttt{SageMath}, ki omogoča učinkovito delo z grafi
ter vsebuje številne vgrajene metode za preverjanje njihovih lastnosti. Najprej bova implementirala
funkcijo \texttt{is\_outerplanar(G)}, ki za dan graf $G$ preveri, ali je outerplanaren. Pri tem
bova uporabila vgrajeno metodo \texttt{G.is\_outerplanar()}, po potrebi pa tudi metode
\texttt{G.is\_planar()} in preverjanje prepovedanih podgrafov $K_4$ in $K_{2,3}$ z uporabo
funkcij \texttt{G.subgraph()} ter \texttt{G.is\_isomorphic(H)}.

Nato bova definirala funkcijo \texttt{alpha\_od(G)}, ki za dan graf $G$ izračuna liho neodvisno
število $\alpha_{od}(G)$. Funkcija bo temeljila na pregledu vseh podmnožic množice vozlišč
ter preverjanju pogojev lihe neodvisnosti. Ker je takšen pristop računsko zahteven, bo uporabljen
le za grafe z manjšim številom vozlišč.

Za sistematično preverjanje majhnih grafov bova uporabila vgrajeno funkcijo
\texttt{graphs(n)}, ki generira vse (neizomorfne) grafe z $n$ vozlišči. Vsak generiran graf bova
najprej testirala na outerplanarnost, nato pa izračunala vrednost $\alpha_{od}(G)$ in preverila,
ali velja neenakost
\[
\alpha_{od}(G) \ge \frac{n}{7}.
\]
Rezultate bova po potrebi shranjevala ter primerjala, da se izogneva podvajanju izomorfnih grafov.

Ker sistematično generiranje grafov hitro postane računsko prezahtevno, bova za večje vrednosti
$n$ uporabila naključno generiranje outerplanarnih grafov. V ta namen bova implementirala
funkcijo, ki najprej generira cikel s pomočjo metode \texttt{graphs.CycleGraph(n)}, nato pa mu
dodaja nekrižajoče se diagonale. Pri tem bova pazila, da ohraniva outerplanarnost grafa.
Več takšnih 2-povezanih outerplanarnih grafov bova nato združila v drevesno strukturo z uporabo
operacij nad vozlišči in povezavami, s čimer bova dobila splošne outerplanarne grafe.

Za dodatno testiranje bova po potrebi uporabila tudi naključne grafe, generirane z metodo
\texttt{graphs.RandomGNP(n,p)}, pri čemer bova obdržala le tiste grafe, ki so outerplanarni.
Za vsak generiran graf bova preverila osnovne lastnosti, kot so število vozlišč
(\texttt{G.order()}), število povezav (\texttt{G.size()}), zaporedje stopenj vozlišč
(\texttt{G.degree\_sequence()}), ter nato izračunala vrednost $\alpha_{od}(G)$.

Na koncu bova rezultate analizirala in primerjala glede na velikost grafa ter strukturo
outerplanarnih grafov. Posebno pozornost bova namenila primerom, pri katerih je vrednost
$\alpha_{od}(G)$ blizu spodnje meje $\frac{n}{7}$, saj bi ti lahko kazali na potencialno
ekstremne ali celo protiprimerne grafe.

\end{document}