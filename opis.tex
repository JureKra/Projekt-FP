\documentclass{article}
\usepackage{amsmath}
\usepackage{amsthm}
\usepackage{amssymb}
\usepackage[slovene]{babel}
\usepackage[utf8]{inputenc}
\usepackage[T1]{fontenc}

\title{Outerplanar graphs}
\author{Jure Kraševec, Urh Videčnik}

\begin{document}
\maketitle

\section{Uvod}
Naj bo $G = (V, E)$ graf, kjer je $V$ množica vozlišč in $E$ množica povezav med vozlišči. Outerplanarni graf (ang. Outherplanar graph) je graf, ki ga lahko narišemo v ravnini tako, 
da se nobeni dve povezavi ne sekata in da vsa vozlišča ležijo na zunanji strani oziroma zunanjem licu grafa. Takšni grafi so podmnožica 
planarnih oziroma ravninskih grafov, ki jih lahko narišemo v ravnini brez presekajočih se povezav, 
vendar vozlišča niso nujno na zunanji strani. Graf je outerplanaren natanko tedaj, ko ne vsebuje podgrafa, 
ki je homeomorfen z $K_4$ ali z $K_{2,3}$. 

\noindent 
Množica vozlišč $S \subseteq V$ grafa $G$ je liha neodvisna množica, če za vsako vozlišče $u \in I$ velja, da je lihe stopnje ter za vsak par $u, v \in I $ velja, da med njima ne obstaja povezava v $E$.
Poleg tega velja, da vsako vozlišče $v \in V \setminus S$ nima nobenega soseda v $S$: $N(v) \cap S = \emptyset$ ali pa ima liho število sosedov v $S$: $|N(v) \cap S| \equiv 1 \pmod{2}$ je liho število.
$N(v)$ označuje množico sosedov vozlišča $v$. Največji moči takšne množice pravimo liho neodvisno število grafa, ki ga označimo z $\alpha_{od}(G)$.


\section{Opredelitev problema}
V nalogi nas bo zanimalo število $\alpha_{od}(G)$. Cilj naloge, je preveriti, če za vsak outerplanaren graf $G$ velja neenakost 
$$\alpha_{od}(G) \geqslant n/7,$$ 
kjer je $n$ število vozlišč grafa $G$. 
Najprej bo potrebno implementirati funkcijo, ki za dan graf preveri, če je outerplanaren. 
Nato bo potrebno razviti algoritem, ki za dan outerplanaren graf izračuna liho neodvisno število $\alpha_{od}(G)$ 
ter preveri, če velja neenakost $\alpha_{od}(G) \geqslant n/7$.
V nadaljevanju bo treba implementirati funkcijo, ki bo generirala naključne outerplanarne grafe različnih velikosti in zanje preverila zgornjo neenakost.
Večje outerplanarne grafe bomo generirali z združevanjem ciklov poljubne dolžine, katerim dodajamo nepresečne diagonale. Dobljene grafe skupaj "zlepimo"
v drevesno strukturo.

\section{Načrt dela}



\end{document}